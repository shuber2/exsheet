% grep begin{exercise} exsheet2-demo-klausur.tex | sed 's/^.*\[\(.*\)\]/\1/' | awk ' { s += $1 } END { print s } '
\documentclass[ngerman,twoside,exam,showanswers]{exsheet2}

\usepackage{fontspec}


\title{4.\ Klausur (Teil 2)}
\course{VO Microcontroller}
\curriculum{ITS-B, ITSB-B}

\date{2022-10-21}
\semester{SS 2022}

\author{Stefan Huber}
\institute{Department IT}
\school{FH Salzburg}
\duration{90 min.}


\instructions{%
\textbf{Hilfsmittel:}
Es ist ein nicht-programmierbarer Taschenrechner erlaubt. Darüber hinaus sind
keine Hilfsmittel gestattet. Mobiltelefon sind auszuschalten und zu verstauen.
Dringender Erreichbarkeit mit der Klausuraufsicht absprechen.

\medskip

\textbf{Hinweise:}
Schreiben Sie leserlich mit Kugelschreiber in blau oder schwarz. Als
Zusatzblätter dürfen ausschließlich ausgegebene abgestempelten Blätter
verwendet werden. Lösen Sie die Klammer nicht und beschriften Sie alle Blätter,
die Sie abgeben, mit Namen und Personenkennzeichen.

\medskip

Antworten Sie präzise und nicht überschießend.

\bigskip

Die Vervielfältigung und Verbreitung, auch auszugsweise, ist nur nach
vorheriger, schriftlicher Zustimmung erlaubt. Die Erstellung von Privatkopien
unterliegt §\,42\ UrhG.

\begin{center}
  Alles Gute!
\end{center}

}



\begin{document}

\maketitle


\begin{exercise}[6]
  Welche Beschleunigung $\vec{a}$ erfährt eine träge Masse $m$ im Kraftfeld
  $\vec{F}$ nach Newton?
  \vspace{2cm}
\end{exercise}


\begin{answer}
  $\vec{a} = \frac{\vec{F}}{m}$
\end{answer}


\begin{exercise}[3]
  Welche Beschleunigung $\vec{a}$ erfährt eine träge Masse $m$ im Kraftfeld
  $\vec{F}$ nach Newton?
  \vspace{2cm}
\end{exercise}


\begin{exercise}[1]
  Welche Beschleunigung $\vec{a}$ erfährt eine träge Masse $m$ im Kraftfeld
  $\vec{F}$ nach Newton?
  \vspace{2cm}
\end{exercise}


\newpage\null
(Some empty page to create some space for student's notes but with exam header.)
\newpage\null

\end{document}
