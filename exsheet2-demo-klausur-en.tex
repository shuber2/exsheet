% grep begin{exercise} exsheet2-demo-klausur-en.tex | sed 's/^.*\[\(.*\)\]/\1/' | awk ' { s += $1 } END { print s } '
\documentclass[twoside,exam,showanswers]{exsheet2}

\usepackage{fontspec}



\title{2.\ Final Exam}
\course{ILV Network-oriented Software Engineering}
\curriculum{ITS-B, ITSB-B, WIN}

\date{21.10.2022}
\semester{SS 2022}

\author{Stefan Huber}
\institute{Department IT}
\school{FH Salzburg}
\duration{90 min.}


\instructions{%
\textbf{Permitted aids:}
A non-programmable calculator is allowed. Beyond that no additional aids are
permitted. Turn off your mobile phone and stow it. In case of urgent reasons of
availability contact the exam supervisor.

\medskip

\textbf{Notices:}
Write readable with a non-erasable pen in blue or black color. If you need
additional paper, use only the provided, stamped sheets of paper. Keep the
stapling intact. Put your name and number on all sheets you hand over.

\medskip

Give precise and concise answers.

\bigskip

It is not permitted to copy or distribute, even only partially, without
explicit, written permit to do so. Making private copies is regulated by
§\,42\ UrhG.

\begin{center}
  All the best!
\end{center}

}



\begin{document}

\maketitle


\begin{exercise}[6]
  Welche Beschleunigung $\vec{a}$ erfährt eine träge Masse $m$ im Kraftfeld
  $\vec{F}$ nach Newton?
  \vspace{2cm}
\end{exercise}

\begin{answer}
  $\vec{a} = \frac{\vec{F}}{m}$
\end{answer}


\begin{exercise}[4]
  Kreuzen Sie die richtigen Anworten an:
  \begin{choices}
    \item $1 + 1 = 0 \pmod 2$
    \item $1 + 1 = 1 \pmod 2$
    \item $1 + 1 = 1 \pmod 3$
    \item $1 + 1 = 5 \pmod 7$
  \end{choices}
\end{exercise}


\begin{exercise}[1]
  Welche Beschleunigung $\vec{a}$ erfährt eine träge Masse $m$ im Kraftfeld
  $\vec{F}$ nach Newton?
  \vspace{2cm}
\end{exercise}


\newpage\null
(Some empty page to create some space for student's notes but with exam header.)
\newpage\null

\end{document}
